\chapter{Conclusion and Future Work}
\label{chap_final}

\section{Conclusion}

With a goal of designing more emotionally engaging robots, I started this work by asking what makes us human. To the relief of all, I did not attempt to answer it over the course of this master's thesis\footnote{The Amalgamated Union of Philosophers, Sages, Luminaries and Other Professional Thinking Persons threatened a strike.}. Instead, I narrowed the question down to an examination of a minimal aspect of living things, namely their ability to grow or change from experience. First, I argued why perceiving a story of experience would engender empathy for a robot. Next, through a series of constructions and human subject studies of robots with increasing complexity, I demonstrated that implicit life stories can evoke empathy. I  also showed that empathy for robots could have an impact on empathy for others. Finally, I recorded my experiences and provided some guidelines to help others design life-story robots. I believe that this work will help bring companion robots one step closer to reality. 

\section{Contributions}


\begin{itemize}
\item Proposed a new design element, implicit life-stories, for engendering empathy for social robots
\item Demonstrated empathy for robots with life-stories through human robot interaction experiments across two different platforms
\item Designed and built a novel sound interaction social robot to embody and test these ideas
\item Demonstrated that the implicit life-stories can improve people's perception of social robot's animacy, anthropomorphism, likeability and intelligence
\item Showed that empathy for robot has an impact on subsequent empathy for humans. This is the first work to examine the connection between the two.
\item Open-sourced design and code for the new sound robot
\end{itemize}


\section{Reflections}

So ends the thesis. I will take off my academic hat and describe why the thesis matters to me. Consider these as opinions, subjective and debatable.

An important challenge when designing companion robots design is being able to support engagement over long term interactions. The novelty of fur, zoomorphic appearance, and life-like movement will wear off.  Functionality such as cleaning or being an interface to connected devices can provide a reason for engagement. However, this functional engagement will be, at best, orthogonal to the relational engagement, or at worst will dwarf the companionship aspect of the robot.

I believe the key to successfully finding companionship with a robot is the ability to have shared experiences with it. However, before we can have meaningful shared experiences, we would have to believe that a robot has the capacity to experience the world in a way similar to us. The main contribution of this work has been to establish the criteria to achieve this very end, with empathy as the measure of success.

 The contribution of the thesis is not just specifying what is required to evoke empathy, but also identifying and deliberately leaving out those things that are not necessary. For instance, the robots in my studies did not need an explicit human life story (eg. saying ``I really miss my parents'') to create an empathic connection. While such stories can be effective, they can also be perceived as inauthentic if the robot is incapable of such experiences. Over the course of this investigation, I also found that an overt appeal to emotions, or even motivation on part of the robot was unnecessary. Emotions and motivations are constructed by the viewer out of their experiences. This is not to say that a robot should not use an affective model. It maybe that an emotion model or even a motivation model will generate behavior that will support projecting an affective model on to the robot. However, there maybe other models available for nurturing the construction of the robot's experience. My hope is that the simplicity of the constraints for implicit life-story will allow robot designers, who adhere to these constraints, to utilize a wide variety of models to guide robot's behavior while still being assured that their robot will engender empathy. 

In addition to developing emotionally engaging robot, this research is a part of a personal journey into understanding what separates ``living'' from ``nonliving''. If we examine the reasons for existence of life, that is chance and evolution, to provide an answer, we would conclude that there is no essential difference between the living and nonliving, both being different approaches to creating stable patterns across time\cite{dawkins_selfish_gene}. If we try to understand the living by creating accurate functional models of processes such as emotions and motivations, we run the risk of objectifying the living and reducing it back to nonliving. To understand the subjective experiences of a living creature, we have to ask not what it is but what we perceive it to be. This means that the task of constructing a life-like machine requires operationalizing this perception so that it is testable, and then developing a set of rules for passing the test. In this research, I used empathy for the test and implicit life-story as a mechanism for passing the test.



\section{Future Work}




% In the discussion of the pilot studies, I talekd about creating the right
% level of ambiguity in relating an experience. One quesiton is how do we 
% design robots that can generate this automatically. As the robot becomes 
% more complex the ambiguity needed will be more nuanced to support projection.
% There are a few ways to do this
% - narrative engines
% - emotional model
% - motivation
% - schmidhuber


One extension of the work is increasing the envelope of experience or the \emph{umwelt} of the robot. The robot in the study was largely confined to experiencing words but as the feedback showed, its perception needs to include a richer sound-scape such as environmental sounds, music, prosody, or accents. Of course, perception could be extended beyond sound to sight and touch. Similarly expression of its experience can be broadened to music, motion, lights or other novel mechanisms.

In the last chapter, I noted that experience of the robot must be perceived to matter to the robot. I discussed how this perception can be supported by ambiguity in communication of the robot's experience. As the robot scales up in complexity of experience, finding the right balance of ambiguity to express the experience but still a support a viewer's projection  will get more challenging. One way would be to implement an computational emotion model \cite{marsella_computational_emotion}. If the robot has artificial but still human like emotions, it should be easier to project emotions on to it. It is worth stressing that the goal for an emotional model is to merely produce the right behavior to communicate an experience and not to suggest thinking of the robot in terms of the model. A more minimal approach would be to use a motivation based cognitive architecture such as Bach's MicroPsi \cite{bach_micropsi}. Emotions would be read into the triumphs and failures of a robot trying to achieve its goals. 

However, is there something more minimal than motivation that could still generate behavior that engage us in relating to the robot? Schmidhuber argues that one of our fundamental drives is seeking out novel patterns that are not too easy but not to difficult to assimilate, that is to say the patterns are \emph{learnable} \cite{schmidhuber_art_science}. Based on this work, one approach could be to generate a pattern in the communication of experiences that make a perceived affective model learnable for a viewer. 

As I discussed in the last section, the work in this thesis is a stepping stone to developing a robot that is perceived to be capable of having shared experiences with a human. By shared experience, I mean both the human and the robot experiencing something together. I believe that would be important for creating companion robots. Subjects in the study expressed interest in such a robot and in particular wanted a robot that they could watch TV with. With close caption data, sentiment analysis and a computational emotion model, I think this would be tractable project and promising way to study shared experience with a robot. 

The number of elderly suffering from loneliness is increasing in the US and around the world. Feeling of loneliness is a social pain that increases physiological aging and risk of mortality \cite{hawkley_loneliness_matters}. The ability to have an emotional connection and shared experience with a robot could lead to development of companion robots that help alleviate loneliness in the elderly. 


